\documentclass{article}
\usepackage{amssymb}
\usepackage{amsmath}
\title{Example \LaTeX\  Document}
\author{Darren Tapp}
\date{\today}
\newcommand{\RR}{{\mathbb R}}

\begin{document}
\maketitle
\LaTeX\  is a markup language that is intended to produce beautiful
mathematics.  We can display an equation;
\[
x+3 = 5.
\]
or mathematics could be in line as $x=2$.

\textbackslash\ usually preceeds a command.  \$ is also a special symbol that
denotes math mode.  One dollar sign for $\sum_{i=1}^n i$ if you want
the text inline.  Two dollar signs if you want to display
$$
\frac{\partial \psi }{\partial t}
$$
However I prefer \textbackslash [ and \textbackslash ] to display equations.

Do you want to define a linear map $L:{\mathbb R}^2\to {\mathbb R}^3$?  We could
could let $L$ be an embedding of ${\mathbb R}^2$ into ${\mathbb R}^3$.
\[
(x,y) \mapsto (x,y,0)
\]
Note I get tired of typing \verb|{\mathbb R}| so I defined a macro above.  I
now can type about $\RR$ all day long.

For some reason I would like to give an example of a matrix.

\[
\begin{bmatrix}
1& 1& 0 \\
0& 1 & 2 \\
0 & 0 & 1
\end{bmatrix}
\begin{bmatrix}
1 \\
2 \\
3
\end{bmatrix}
=
\begin{bmatrix}
3 \\
8 \\
3
\end{bmatrix}
\]
This document uses two packages with the \verb|\usepackage| declaration above.
\verb|amssymb| is used for the blackboard bold $\RR$.  \verb|amsmath| is used
to make the matrices easier to type.



\end{document}
